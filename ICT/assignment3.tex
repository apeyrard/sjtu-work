\documentclass[10pt]{article}
\author{Alex Peyrard}
\title{Information and coding theory assignment 3}

\begin{document}
\maketitle

\section{Inequalities}
Given $U\rightarrow X\rightarrow Y\rightarrow V$ a Markov chain, we can find :
\[I(U,V)\leq I(U,Y)\]
\[I(U,V)\leq I(X,V)\]

\[I(U,V)\leq I(U,X)\]
\[I(U,V)\leq I(X,Y)\]
\[I(U,V)\leq I(Y,V)\]

\[I(U,Y)\leq I(U,X)\]
\[I(U,Y)\leq I(X,Y)\]

\[I(X,V)\leq I(X,Y)\]
\[I(X,V)\leq I(Y,V)\]

\section{Fano's inequality}

\section{Kraft's inequality}
We want to prove that for a uniquely decodable code, then $\sum\limits_{k=1}^{m}D^{-l_{k}}\leq 1$
With $D$ the size of the size of the alphabet, $m$ the number of words, and $l_{k}$ the length of the $k^{th}$ word.

For the code to be uniquely decodable, it can't have a code that is the beginning of a following code. For a D-ary code, this means that if there are $D$ words of code length $l$, there can't be words of length superior to $l$.

Thus, for the D-ary code to be uniquely decodable, we can at most have $D-1$ words of code length $l < l_{max}$, and $D$ words of code length $l_{max}$.

This means that we have :
\[\sum\limits_{k=1}^{m}D^{-l_{k}}=(D-1)\sum\limits_{i=1}^{l_{max-1}}D^{-i}+D*D^{-l_{max}}\]
\[\sum\limits_{k=1}^{m}D^{-l_{k}}=(D-1)\sum\limits_{i=1}^{l_{max}}D^{-i}+D^{-l_{max}}\]
\[\sum\limits_{k=1}^{m}D^{-l_{k}}=(D-1)\frac{D^{-l_{max}}(D^{l_{max}-1})}{D-1}+D^{-l_{max}}\]
\[\sum\limits_{k=1}^{m}D^{-l_{k}}=1\]
Since this is at the most, a code is uniquely decodable if
\[\sum\limits_{k=1}^{m}D^{-l_{k}}\leq1\]
QED

\end{document}