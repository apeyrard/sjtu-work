\documentclass[10pt]{article}
\author{Alex Peyrard}
\title{Digital Image Processing}
\usepackage{graphicx}

\begin{document}
\maketitle
\section{Introduction}
All of the programs are written in python 3. The shebangs are included and thus they should work if called using the "./program.py" notation. In case this causes a problem, please try to call them using "python3 program.py" notation.\\\\
I will provide further help on how to call each program.
\subsection{libraries}
I used numpy in all of the programs and scipy in some of them. I will detail when scipy is used.
\section{Exercise 1}
The program called ex1.py computes the histogram of a greyscale image, and enhances the image using histogram equalization. it displays the enhanced image, the histograms of the default and enhanced images, and the transformation function.\\
In this exercise, the matplotlib library is used to plot hitograms and functions.
\subsection{Examples}
\subsubsection{Fig1.jpg}
All of the results are obtained using the program with the call "./ex1.py Fig1.jpg"
\begin{figure}[!ht]
	\centering
	\includegraphics[height=200pt]{./ex1/Fig1.jpg}
	\caption{Original image}
\end{figure}
\begin{figure}[!ht]
	\centering
	\includegraphics[height=200pt]{./ex1/Fig1_hist.png}
	\caption{Original image's histogram}
\end{figure}
\begin{figure}[!ht]
	\centering
	\includegraphics[height=200pt]{./ex1/Fig1_cdf.png}
	\caption{Enhancement function}
\end{figure}
\begin{figure}[!ht]
	\centering
	\includegraphics[height=200pt]{./ex1/Fig1_enh_hist.png}
	\caption{Enhanced image histogram}
\end{figure}
\begin{figure}[!ht]
	\centering
	\includegraphics[height=200pt]{./ex1/Fig1_enh.jpg}
	\caption{Enhanced image}
\end{figure}
\clearpage

\subsubsection{Fig2.jpg}
All of the results are obtained using the program with the call "./ex1.py Fig2.jpg"
\begin{figure}[!ht]
	\centering
	\includegraphics[height=200pt]{./ex1/Fig2.jpg}
	\caption{Original image}
\end{figure}
\begin{figure}[!ht]
	\centering
	\includegraphics[height=200pt]{./ex1/Fig2_hist.png}
	\caption{Original image's histogram}
\end{figure}
\begin{figure}[!ht]
	\centering
	\includegraphics[height=200pt]{./ex1/Fig2_cdf.png}
	\caption{Enhancement function}
\end{figure}
\begin{figure}[!ht]
	\centering
	\includegraphics[height=200pt]{./ex1/Fig2_enh_hist.png}
	\caption{Enhanced image histogram}
\end{figure}
\begin{figure}[!ht]
	\centering
	\includegraphics[height=200pt]{./ex1/Fig2_enh.jpg}
	\caption{Enhanced image}
\end{figure}
\clearpage

\section{Exercise 2}
The program called ex2.py performs several spatial enhancement techniques on a given greyscale image.
\subsection{Example on skeleton\_orig.tif}
All of the results are obtained using the program with the call "./ex2.py skeleton\_orig.tif"
\begin{figure}[!ht]
	\centering
	\includegraphics[height=200pt]{./ex2/skeleton_orig.jpg}
	\caption{Original image}
\end{figure}
\begin{figure}[!ht]
	\centering
	\includegraphics[height=200pt]{./ex2/skeleton_lap.jpg}
	\caption{Rescaled laplacian of image}
\end{figure}
\begin{figure}[!ht]
	\centering
	\includegraphics[height=200pt]{./ex2/skeleton_lap_plus_orig.jpg}
	\caption{Sum of laplacian and original image}
\end{figure}
\begin{figure}[!ht]
	\centering
	\includegraphics[height=200pt]{./ex2/skeleton_sobel.jpg}
	\caption{Sobel gradient of original image}
\end{figure}
\begin{figure}[!ht]
	\centering
	\includegraphics[height=200pt]{./ex2/skeleton_smmoth_sobel.jpg}
	\caption{Smoothed Sobel gradient of original image}
\end{figure}
\begin{figure}[!ht]
	\centering
	\includegraphics[height=200pt]{./ex2/skeleton_product.jpg}
	\caption{Product of smoothed Sobel gradient and of the previous sum of laplacian and original}
\end{figure}
\begin{figure}[!ht]
	\centering
	\includegraphics[height=200pt]{./ex2/skeleton_final.jpg}
	\caption{Sum of original image and previous image}
\end{figure}
\begin{figure}[!ht]
	\centering
	\includegraphics[height=200pt]{./ex2/skeleton_final.jpg}
	\caption{Power law of original image, gamma = 0.5, c = 1}
\end{figure}

\clearpage

\section{Exercise 3}

\clearpage

\end{document}